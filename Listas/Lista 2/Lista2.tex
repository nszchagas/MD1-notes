\documentclass[13pt]{article}
\usepackage[utf8]{inputenc}
\usepackage[portuguese]{babel}
\usepackage{amsmath}
\usepackage{amssymb}

\title{Lista 2 de Matemática Discreta}
\author{Nicolas Chagas Souza}	
\date{\today}

\setlength{\parindent}{0em}
\setlength{\parskip}{1em}
\renewcommand{\baselinestretch}{0.5}




\begin{document}
\maketitle



\begin{enumerate}

	\item[1)] 
   \end{enumerate}
   
    \begin{enumerate}
    \item[a)] $\sum\limits_{i=1}^{n} i^3 = \frac{n^2(n+1)^2}{4}  $
	\end{enumerate}
	Hipótese indutiva: 
	\begin{center}
	$1^3=\frac{1^2(1+1)^2}{4} $
	$1 = \frac{4}{4} $ 
	$1=1$
	\end{center}	
	Então a propriedade é válida para $i = 1$.	
	Suponha agora que a propriedade seja válida para $i = k$, temos:
	
	$\sum\limits_{i=1}^{k} i^3 = \frac{k^2(k+1)^2}{4}  $
	
	Precisamos verificar se a propriedade é válida para $k+1$:
	\begin{center}
	
	
	$\sum\limits_{i=1}^{k+1} i^3 $ 

	
	$\sum\limits_{i=1}^{k} i^3 + (k+1)^3$
	\end{center}
	Pela hipótese indutiva:
	\begin{center}
	$ \frac{k^2(k+1)^2}{4} + (k+1)^3 $ 
	
	$ \frac{k^2(k+1)^2+4(k+1)^3}{4}  $

	$ \frac{k^2(k+1)^2+4(k+1)^2(k+1)}{4} $ 
	
	$ \frac{k^2(k+1)^2+4(k+1)^2(k+1)}{4} $ 
	
	$ \frac{(k+1)^2 (k^2 + 4(k+1))}{4} $
	
	$ \frac{(k+1)^2 (k^2 + 4k +4))}{4} $
	
	$ \frac{(k+1)^2 (k+2)^2}{4} $
	
	$ \frac{(k+1)^2 (k+2)^2}{4} $
	
	$ \therefore \sum\limits_{i=1}^{k+1} i^3  = \frac{(k+1)^2 (k+2)^2}{4} $	
	
	\end{center}
	
	
	
	
	
	
	
	

\end{document}